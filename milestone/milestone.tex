\documentclass{article} % For LaTeX2e
\usepackage{nips13submit_e,times}
\usepackage{hyperref}
\usepackage{url}
%\documentstyle[nips13submit_09,times,art10]{article} % For LaTeX 2.09


\title{Formatting Instructions for NIPS 2013}


\author{
Matthew Long, Desmond C. Ong, Shane Soh \\
Stanford University \\
\texttt{\{mlong14, dco, shanesoh\}@stanford.edu}
%Matthew Long \\
%Stanford University\\
%\texttt{mlong14} \\
%\And
%Desmond C. Ong \\
%%Psychology \\
%Stanford University \\
%\texttt{dco} \\
%\And
%Shane Soh \\
%%Affiliation \\
%Stanford University \\
%\texttt{shanesoh} \\
}

% The \author macro works with any number of authors. There are two commands
% used to separate the names and addresses of multiple authors: \And and \AND.
%
% Using \And between authors leaves it to \LaTeX{} to determine where to break
% the lines. Using \AND forces a linebreak at that point. So, if \LaTeX{}
% puts 3 of 4 authors names on the first line, and the last on the second
% line, try using \AND instead of \And before the third author name.

\newcommand{\fix}{\marginpar{FIX}}
\newcommand{\new}{\marginpar{NEW}}

\nipsfinalcopy % Uncomment for camera-ready version

\begin{document}


\maketitle

\begin{abstract}
The abstract paragraph should be indented 1/2~inch (3~picas) on both left and
right-hand margins. Use 10~point type, with a vertical spacing of 11~points.
The word \textbf{Abstract} must be centered, bold, and in point size 12. Two
line spaces precede the abstract. The abstract must be limited to one
paragraph.
\end{abstract}

\section{Introduction}


This section introduces your problem, and the overall plan for approaching your problem

\section{Problem Statement}
Describe your problem precisely specifying the dataset to be used, expected results and evaluation


\section{Technical Approach and Models} 

Describe the methods you intend to apply to solve the given problem

\section{Intermediate/Preliminary Experiments \& Results}

State and evaluate your results up to the milestone


%\begin{figure}[h]
%\begin{center}
%%\framebox[4.0in]{$\;$}
%\fbox{\rule[-.5cm]{0cm}{4cm} \rule[-.5cm]{4cm}{0cm}}
%\end{center}
%\caption{Sample figure caption.}
%\end{figure}

%\begin{table}[t]
%\caption{Sample table title}
%\label{sample-table}
%\begin{center}
%\begin{tabular}{ll}
%\multicolumn{1}{c}{\bf PART}  &\multicolumn{1}{c}{\bf DESCRIPTION}
%\\ \hline \\
%Dendrite         &Input terminal \\
%Axon             &Output terminal \\
%Soma             &Cell body (contains cell nucleus) \\
%\end{tabular}
%\end{center}
%\end{table}


\subsubsection*{Acknowledgments}



\subsubsection*{References} % chronological. any citation style is acceptable, as long as it's consistent. Use APA

\small{
%[1] Alexander, J.A. \& Mozer, M.C. (1995) Template-based algorithms
%for connectionist rule extraction. In G. Tesauro, D. S. Touretzky
%and T.K. Leen (eds.), {\it Advances in Neural Information Processing
%Systems 7}, pp. 609-616. Cambridge, MA: MIT Press.

%[2] Bower, J.M. \& Beeman, D. (1995) {\it The Book of GENESIS: Exploring
%Realistic Neural Models with the GEneral NEural SImulation System.}
%New York: TELOS/Springer-Verlag.

%[3] Hasselmo, M.E., Schnell, E. \& Barkai, E. (1995) Dynamics of learning
%and recall at excitatory recurrent synapses and cholinergic modulation
%in rat hippocampal region CA3. {\it Journal of Neuroscience}
%{\bf 15}(7):5249-5262.
}

\end{document}
